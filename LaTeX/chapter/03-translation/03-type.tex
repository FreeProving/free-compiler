\section{Translation of Type Expressionss} \label{sec:translation:type}
A computation in Haskell does not necessarily produce a value but can also result in an effect, for example due to partiality.
Coq on the other hand does not support such implicit effects.
As dicussed before, we bypass this restriction by modeling effects using the free monad in Coq.
This fact needs to be reflected on type level as well.
Therefore, every Haskell type $\tau$ is lifted into the free monad during the translation.

\[
  \toCoq{\tau} = \Free{\tau'}
\]

However, the translation rule above is incomplete when we take higher-order functions into account.
Consider for example the following function declaration.
\begin{minted}{haskell}
  map :: (a -> b) -> [a] -> [b]
  map f xs = {- ... -}
\end{minted}

The parameter \haskell{xs} is of type $[a]$ and does not pose any problem.
Assuming \coq{List} is the Coq type constructor corresponding to \haskell{[]} (i.e. $[]' = List$), we can translate the type of \haskell{xs} as follows.
\[
  \toCoq{[a]} = \Free{(\texttt{List}\;a)}
\]
This allows an application of \haskell{map} to pass an effectful computation as the second argument.
If we define the constructors of \coq{List} appopriatly (see \autoref{sec:translation:type-decl}), the list items can also have effects themselves.
We do not have to lift the type argument $a$ explicitly.

The parameter \haskell{f} is different though.
While it is correct to lift its type $a \rightarrow b$ into the free monad such that potential effects in the first argument of \haskell{map} can be handled, this is not sufficent.
\[
  \toCoq{(a \rightarrow b)} = \Free{(a \rightarrow b)}
\]
The argument in an application of \haskell{f} could have an effect as well and its result is also potentially effectful.
In case of \haskell{xs} the \coq{List} constructors took care of lifting the item types.
However, we do not provide a type constructor for \coq{->} that does the same for functions.
Therefore, we need to lift the argument and return type of \haskell{f} explicitly.
\[
  \toCoq{(a \rightarrow b)} = \Free{(\Free{a} \rightarrow \Free{b})}
\]

This needs to be done for all arbitrarily deeply nested function types in $\tau$.
For this purpose we introduce another translation operation $\toCoqT{\tau}$ as shown in \autoref{fig:translation:type}.
\begin{figure}[H]
  \begin{align}
       \toCoq{\tau}
    &= \Free{\toCoqT{\tau}}
    && \forall \tau \text{ type expression}
       \label{eqn:translation:type}
  \end{align}
  \begin{align}
       \toCoqT{(\tau_1 \rightarrow \tau_2)}
    &= \toCoq{\tau_1} \rightarrow \toCoq{\tau_2}
    && \forall \tau_1, \tau_2 \text{ type expression}
       \label{eqn:translation:type:function}
    \\
       \toCoqT{(\tau_1\;\tau_2)}
    &= \toCoqT{\tau_1}\;\toCoqT{\tau_2}
    && \forall \tau_1, \tau_2 \text{ type expression}
       \label{eqn:translation:type:application}
    \\
       \toCoqT{D}
    &= \genApply{D'}
    && \forall D \text{ type constructor}
       \label{eqn:translation:type:constructor}
    \\
    \toCoqT{\alpha}
    &= \alpha'
    && \forall \alpha \text{ type variable}
       \label{eqn:translation:type:variable}
  \end{align}
  \caption{Final translation rules for type expressions.}
  \label{fig:translation:type}
\end{figure}

It recursively lifts the argument and return types of function types contained in $\tau$.
Additionally, the parameters $Shape$ and $Pos$ are passed to type constructors.
That is, in the example above, the type of \haskell{xs} would be translated as follows.
\[
  \toCoq{[a]} = \Free{(\genApply{\texttt{List}}\;a)}
\]
\coq{List} must be provided with $Shape$ and $Pos$ to lift the argument types of its constructors.
Otherwise list items would not be allowed to be effectful.
The translation of data types and their constructors will be elaborated on in the next section.
