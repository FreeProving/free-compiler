\section{Base Library} \label{sec:implementation:base-library}
The code generated by the translation rules presented in \autoref{chp:translation} relies on a definition of the free monad to be available in Coq.
Furthermore, implementations of predefined data types and operations as described in \autoref{sec:preliminaries:assumptions:prelude} must be provided.
For this reason, our compiler is accompanied by a Coq library called \coq{Base}.
The compiler adds a command to the top of every generated Coq file that imports the \coq{Free} and \coq{Prelude} modules of the base library.
\begin{minted}{coq}
  From Base Require Import Free Prelude.
\end{minted}

While the \coq{Prelude} module contains predefined data types and operations, the \coq{Free} module exports definitions of the free monad, bind operator and \coq{Partial} type class.
Concrete instances for \coq{Identity}, \coq{Maybe} and \coq{Error} are included in the Base library as well but not exported by default.

Similar to how user-defined data types sometimes need to be renamed, the symbols used for predefined data types and operations must be given a valid Coq identifier.
For example, the nullary tuple is called \coq{Unit} in the base library while it is simply denoted \haskell{()} in Haskell.
To reduce the coupling between the compiler and the base library, these names are not hard-coded into the compiler but configurable.
For this purpose, there is a \path{env.toml} configuration file in the root directory of the base library whose format is documented in \autoref{appendix:config}.
The separation of the base library from the actual compiler makes it easy to extend the \coq{Prelude} in the future.

One consequence of the separation is, that the user has to explicitly specify the location of the base library using the \coq{--base-libray} command line option (see also \autoref{appendix:usage}) if it cannot be found in the directory where Cabal usually places data files.
The compiler automatically creates a \path{_CoqProject} file in the output directory if it does not exists.
The \path{_CoqProject} tells Coq where the \coq{Base} library can be found.
