\section{Architecture} \label{sec:implementation:architecture}
The Haskell to Coq compiler presented in this thesis is written in Haskell itself.
This allows us to make use of Haskell's package infrastructure.
For example, we do not have to implement our own parser for Haskell modules.
To manage dependencies and to build the compiler, we are using Cabal\footnote{\url{https://www.haskell.org/cabal/}}.
Instructions for how to setup, build and run the compiler can be found in \autoref{appendix:usage}

\begin{figure}[H]
  \digraph[scale=0.75]{CompielerArchitecture}{
    node[shape="rect"];
    edge[minlen=1];
    rankdir="LR";

    # Left nodes.
    \{
      rank="same";
      node[width=2];
      input[label="Haskell source file"];
      haskellAst[label="Haskell AST"];
      simpleAst[label="Simplified AST"];
    \}

    # Right nodes.
    \{
      rank="same";
      node[width=3];
      output[label="Coq source file"];
      coqAst[label="Coq AST"];
      scc[label="Strongly Connected Components"];
    \}

    # Invisible edges between opposite facing nodes.
    input      -> output [style="invis"];
    haskellAst -> coqAst [style="invis"];

    # Vertical edges pointing down.
    input      -> haskellAst [xlabel="Parser&nbsp;&nbsp;&nbsp;"];
    haskellAst -> simpleAst  [xlabel="Simplifier&nbsp;&nbsp;&nbsp;"];

    # Horizontal edge.
    simpleAst  -> scc        [label="Dependency Analysis"];

    # Vertical edges pointing up.
    coqAst     -> scc        [dir="back",xlabel="Converter&nbsp;&nbsp;&nbsp;"];
    output     -> coqAst     [dir="back",xlabel="Pretty Printer&nbsp;&nbsp;&nbsp;"];
  }
  \caption{...}
  \label{fig:implementation:architecture}
\end{figure}
