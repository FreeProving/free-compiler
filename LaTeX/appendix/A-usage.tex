\chapter{Installation and Usage} \label{appendix:usage}

\paragraph{Directory Structure}
The implementation of the Haskell to Coq compiler presented in this thesis can be found in the \path{haskellToCoqCompiler} subdirectory of our Git repository\footnote{\url{https://git.informatik.uni-kiel.de/stu203400/bachelor-thesis}}.
The code is structured as follows.
\begin{itemize}
  \item \path{base}: Contains the Coq base library.
    \begin{itemize}
      \item \path{Free}: Contains Coq files for the definition of the free monad.
      \item \path{Prelude} Contains Coq files for predefined data types and operations.
      \item \path{env.toml} Configuration file (see \autoref{appendix:config}) that contains the names of predefined data types and operations.
    \end{itemize}
  \item \path{example}: Contains example Haskell and Coq code.
    \begin{itemize}
      \item \path{manual}: Manually translated Coq files.
      \item \path{generated}: Coq files generated by the compiler (not included in distributed source).
      \item \path{ExampleQueue.hs} The input module used in the case study.
      \item \path{ExampleQueueTests.v} Proofs and lemmas for QuickCheck properties in \path{ExampleQueue.hs}.
    \end{itemize}
  \item \path{src} Haskell source code.
  \item \path{tool} Bash scripts for running and testing the compiler during development.
\end{itemize}

\paragraph{Required Software}
The Haskell to Coq compiler is written in Haskell and uses Cabal to manage its dependencies.
To build the compiler, the GHC\footnote{\url{https://www.haskell.org/ghc/}} and Cabal\footnote{\url{https://www.haskell.org/cabal/}} are required.
To use the Coq code generated by our compiler, the Coq proof assistant\footnote{\url{https://coq.inria.fr/download}} must be installed.
The compiler has been tested with the following software versions on a Debian based operating system.
\begin{itemize}
  \item GHC, version 8.6.5
  \item Cabal, version 2.4.1.0
  \item Coq, version 8.8.2
\end{itemize}

\paragraph{Compiling the Base Library}
In order to use the base library, the Coq files in the base library need to be compiled first.
Make sure to compile the base library \textbf{before} installing the compiler.
We provide a shell script for the compilation of Coq files.
To compile the base library with that shell script, run the following command in the root directory of the compiler.
\begin{minted}{bash}
  ./tool/compile-coq.sh base
\end{minted}

\paragraph{Installation}
First, make sure that the Cabal package lists are up to date by running the following command.
\begin{minted}{bash}
  cabal new-update
\end{minted}
To build and install the compiler and its dependencies, change into the compiler's root directory and run the following command.
\begin{minted}{bash}
  cabal new-install haskell-to-coq-compiler
\end{minted}
The command above copies the base library and the compiler's executable to Cabal's installation directory and creates a symbolic link to the executable in \path{~/.cabal/bin}.
To test whether the installation was successful, make sure that \path{~/.cabal/bin} is in your \bash{PATH} environment variable and run the following command.
\begin{minted}{bash}
  haskell-to-coq-compiler --help
\end{minted}

\paragraph{Running without Installation}
If you want to run the compiler without installing it on your machine, execute the following command in the root directory of the compiler instead of the \bash{haskell-to-coq-compiler} command.
\begin{minted}{bash}
  cabal new-run haskell-to-coq-compiler -- [options...] <input-files...>
\end{minted}
The two dashes are needed to separate the arguments to pass to the compiler from Cabal's arguments.
Alternatively, you can use the \path{./tool/run.sh} bash script.
\begin{minted}{bash}
  ./tool/run.sh [options...] <input-files...>
\end{minted}

\paragraph{Usage}
To compile a Haskell module, pass the file name of the module to \bash{haskell-to-coq-compiler}.
For example, to compile the example module from the case study, run the following command.
\begin{minted}{bash}
  haskell-to-coq-compiler ./example/ExampleQueue.hs
\end{minted}
The generated Coq code is printed to the console.
To write the generated Coq code into a file, specify the output directory using the \bash{--output} (or \bash{-o}) option.
For example, the following command creates a file \path{example/generated/ExampleQueue.v}.
\begin{minted}{bash}
  haskell-to-coq-compiler -o ./example/generated ./example/ExampleQueue.hs
\end{minted}
In addition to the Coq file, a \path{_CoqProject} file is created in the output directory if it does not exist already.
The \path{_CoqProject} file tells Coq where to find the compiler's base library.
Add the \bash{--no-coq-project} command line flag to disable the generation of a \path{_CoqProject} file.

In order to compile Haskell modules successfully, the compiler needs to know the names of predefined data types and operations.
For this purpose, the \path{base/env.toml} configuration file has to be loaded.
If the compiler is installed as described above, it will be able to locate the base library automatically.
Otherwise, it may be necessary to tell the compiler where the base library can be found using the \bash{--base-library} (or \bash{-b}) option.

Use the \bash{--help} (or \bash{-h}) option for more details on supported command line options.
