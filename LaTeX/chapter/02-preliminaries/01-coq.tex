\section{The Proof Assistant Coq} \label{sec:preliminaries:coq}
Coq is an interactive theorem prover that has been under development since 1983.
Unlike automated theorem provers, Coq does not attempt to prove a proposition autonomously but in an interactive fashion with the help of human guidance.
First, the user creates a model of the mathematical objects and algorithms they want to reason about in a functional programming language called \textit{Gallina}.
The same language is then used to state propositions about the functional program and construct proofs for these propositions.
Coq offers so called \textit{tactics} to automate some otherwise tedious tasks in the proof construction process.
The correctness of the final proof is then checked by Coq.
Hereinafter, we are going to use the term Coq to refer to both the proof assistant and the programming language.
In the following subsections, the most important syntactic constructs of Coq's specification language used throughout this thesis are presented.

\subsection{Types and Data Type Declarations} \label{sec:preliminaries:coq:data-decls}
In contrast to many other programming languages, Coq has a minimal set of build-in data types.
Data types like numbers and booleans~--~which are commonly directly build into a language~--~can be defined using the same mechanisms as for user-defined types.
Custom types are declared using the \coq{Inductive} keyword.
The following example demonstrates the declaration of a boolean data type.
\begin{minted}{coq}
  Inductive bool : Type
    := true
     | false.
\end{minted}
In Coq, declarations are formed by so called \textit{sentences} which start with a capitalized \textit{command} and end with a period \cite[p.~30]{CoqDevTeam:2018}.
The \coq{Inductive} sentence, in the example above, declares a data type \coq{bool} with two constructors \coq{true} and \coq{false}.
Unlike in Haskell, there are no capitalization rules for identifiers.
The colon in the first line annotates the type of \coq{bool}.
In this case, \coq{bool} is of the build-in type \coq{Type}.
Similarly, we can use a colon to declare the type of the constructors.
Both \coq{true} and \coq{false} are constants of the newly declared type \coq{bool}.
\begin{minted}{coq}
  Inductive bool : Type
    := true  : bool
    |  false : bool.
\end{minted}
Natural numbers can be represented in Coq by the means of Peano numbers using the following \coq{Inductive} sentence.
\begin{minted}{coq}
  Inductive nat : Type
    := O           : nat
    |  S (n : nat) : nat.
\end{minted}
This time, the constructor \coq{S} takes an argument \coq{n} of type \coq{nat}.
\coq{(n : nat)} is called a binder for the argument \coq{n}.
Multiple arguments of the same type can be combined into a single binder.
For example, \coq{(n m : nat)} declares two arguments \coq{n} and \coq{m}  of type \coq{nat}.
In constructor declarations we can also choose to declare the arguments anonymously.
In this case, the constructor \coq{S} must be annotated with a function type as shown below.
\begin{minted}{coq}
  Inductive nat : Type
    := O : nat
    |  S : nat -> nat.
\end{minted}
Even though, they are not directly build into the language, the types \coq{bool} and \coq{nat} are part of Coq's standard library and can be used in every Coq program.

Coq also supports the declaration of polymorphic types.
Type arguments are passed like regular arguments, i.e., to declare a polymorphic list data type, we just have to add a binder for a type variable to the \coq{Inductive} sentence.
\begin{minted}{coq}
  Inductive list (X : Type) : Type
    := nil  : list X
    |  cons : X -> list X -> list X.
\end{minted}
Now \coq{list} is a function that produces a list data type for the given value data type.
In Coq, we can print the type of a term using the \coq{Check} command.
\begin{minted}{coq}
  Check list. (* ==> list : Type -> Type *)
\end{minted}
However, the type argument \coq{X} has been added to the constructors \coq{nil} and \coq{cons} as well.
\begin{minted}{coq}
  Check nil.  (* ==> nil  : forall X : Type, list X *)
  Check cons. (* ==> cons : forall X : Type, X -> list X -> list X *)
\end{minted}
We can interpret types with \coq{forall} quantifiers as function types which additionally give a name to their arguments.
In other words, \coq{nil} is a function which takes a type as its first argument \coq{X}.
The return type of that function depends on the type given as \coq{nil}'s first argument.
The term \coq{nil nat} for example is of type \coq{list nat}.
Since the return types of functions can depend on their arguments, Coq is called a \textit{dependently typed language}.

We do not have to pass the type arguments explicitly though.
Using a \coq{Arguments} sentence we can specify which arguments of a function should be passed explicitly and which should be inferred by Coq.
Inferred arguments are simply wrapped in curly braces.
\begin{minted}{coq}
  Arguments nil {X}.
  Arguments cons {X}.
\end{minted}
Now, the type of \coq{nil} is \coq{list ?X} where \coq{?X} is a type inferred from the respective context.
It is also possible to specify that the arguments bound by a binder should be inferred in the declaration itself.
The following marks the type argument \coq{X} as implicit for example.
\begin{minted}{coq}
  Inductive list {X : Type} : Type
    := nil  : list X
    |  cons : X -> list X -> list X.
\end{minted}
However, now \coq{X} is not only an implicit argument of \coq{nil} and \coq{cons} but also of \coq{list}.
Therefore, \coq{list} tries to infer the value type from the context which is not necessarily what we want.

\subsection{Function Declarations and Expressions} \label{sec:preliminaries:coq:func-decls}
In Coq functions are declared using \coq{Definition} sentences.
The \coq{Definition} command is followed by the name of the function to declare, binders for its arguments, an annotation of the function's return type and finally a term on the right-hand side of the function declaration.
For example, the following \coq{Definition} sentence declares a function \coq{null} that tests whether a list \coq{xs} is empty or not.
\begin{minted}{coq}
  Definition null {X : Type} (xs : list X) : bool :=
    match xs with
    | nil        => true
    | cons x xs' => false
    end.
\end{minted}
On the right-hand side of \coq{null} we are using a \coq{match} expression to perform pattern matching.
The \coq{match} expression corresponds to a \haskell{case} expression in Haskell.
However, Coq requires us to list all constructors of the matched data type.
Therefore, the following declaration of a function that returns the first element of a list is not allowed in Coq since there is no alternative for the \coq{nil} constructor.
\begin{minted}{coq}
  Fail Definition head {X : Type} (xs : list X) : X :=
    match xs with
    | cons x xs' => x
    end.
  (* ==> The command has indeed failed with message:
         Non exhaustive pattern-matching: no clause found for pattern nil *)
\end{minted}
The \coq{Fail} command checks that the subsequent declaration contains an error and then continues processing the Coq source file without declaring the function.

Further kinds of Coq expressions known from Haskell include \coq{if} expressions, number literals and lambda abstractions.
\coq{if} expressions look just like in Haskell but since \coq{bool} is not build directly into Coq, they work with any data type which has exactly two constructors \cite[p.~48]{CoqDevTeam:2018}.
The following example demonstrates, that we can use values of type \coq{nat} for the condition of \coq{if} expressions.
The first constructor \coq{O} is handled like \coq{true} and the second constructor \coq{S} like \coq{false}.
\begin{minted}{coq}
  Compute if O then 1 else 2.   (* ==> 1 *)
  Compute if S O then 1 else 2. (* ==> 2 *)
\end{minted}
The literals \coq{1} and \coq{2} denote values of type \coq{nat} and are short for \coq{S O} and \coq{S (S O)}, respectively.

Lambda abstractions are created with the \coq{fun} keyword followed by binders for the arguments.
The following lambda abstraction computes the successor of a natural number.
\begin{minted}{coq}
  fun (n : nat) => S n
\end{minted}
Similar to how we did not have to annotate the types of \coq{x} and \coq{xs'} in the pattern \coq{cons x xs'} above, we do not have to annotate the type of the lambda abstraction's argument either.
\begin{minted}{coq}
  fun n => S n
\end{minted}
In this case, Coq infers the type of \coq{n} from the right-hand side of the lambda abstraction.

One crucial difference between functions in Haskell and Coq is, that Coq distinguishes between non-recursive functions~--~like \coq{null}~--~and recursive functions.
Recursive functions must be declared with the \coq{Fixpoint} command instead of the \coq{Definition} command.
The following function computes the length of a list for example.
\begin{minted}{coq}
  Fixpoint length {X : Type} (xs : list X) : nat :=
    match xs with
    | nil        => O
    | cons x xs' => S (length xs')
    end.
\end{minted}
Coq employs an additional check to verify that the function declarations declared by such \coq{Fixpoint} sentences terminate on all inputs.
More details on the Coq's termination analysis are given in \autoref{sec:translation:func-decl:rec}.

\subsection{Propositions and Proofs} \label{sec:preliminaries:coq:theorems}
Since Coq is a proof assistant, it provides a mechanism to denote propositions and their proofs.
Propositions are given a name through \coq{Theorem} sentences.
For example, the following sentence declares a theorem called \coq{null_length} that states that if a list is empty, its length is zero.
\begin{minted}{coq}
  Theorem null_length: forall (X : Type) (xs : list X),
                       null xs = true -> length xs = 0.
  Proof. (* ... *) Qed.
\end{minted}
After the theorem has been declared, the user writes a proof script which constructs a proof for the theorem using the tactics provided by Coq.
Since we will not write any proofs using Coq in this thesis, the proof script is left blank in the example above.
However, one aspect that we have to understand about proofs in Coq is how Coq checks whether the proof constructed by the user actually proves the stated proposition.

Coq makes use of the so called \textit{Curry-Howard isomorphism}, to check the validity of proofs \cite[p.~1]{CoqDevTeam:2018}.
Basically, propositions are represented by types and values correspond to proofs.
A proposition is true if and only if there is a value of the corresponding type.
Therefore, Coq can employ its type checker to verify that the proof specified by the user is correct.
For example, the propositions \coq{True} and \coq{False} are defined as follows in Coq.
\begin{minted}{coq}
  Inductive False : Prop := .
  Inductive True  : Prop := I : True.
\end{minted}
The build-in type \coq{Prop} denotes the type of propositions and is a subset of \coq{Type}.
Hence, \coq{False} is a type with no constructor and \coq{True} is a type with exactly one value denoted by the constant \coq{I}.
A theorem that states that the proposition \coq{True} is true just needs to apply the constructor \coq{I} to create a value of type \coq{True} which serves as evidence, that the proposition holds.
\begin{minted}{coq}
  Theorem true_is_true: True.
  Proof. apply I. Qed.
\end{minted}
On the other hand, since no value of type \coq{False} exists, the proposition \coq{False} does not hold which is exactly what we would expect.

Of course, the proof scripts of \coq{null_length} and other real theorems and the proofs constructed by them are much more complicated than that.
What may be confusing is that the proposition on the right-hand side of \coq{null_length} does not even look like a type anymore since it contains values like \coq{true} and \coq{0}.
That values occur within types is a common feature of dependently typed languages.
In fact, there is no difference between types and terms in Coq.
As we have seen already, types are passed to functions like regular arguments for example.

\subsection{Differences between Haskell and Coq} \label{sec:preliminaries:coq:summary}
Lets conclude this brief introduction to Coq by summarizing the key differences between Haskell and Coq.
First of all, both Coq and Haskell are functional programming languages and share a lot of similarities as such.
Additionally, Coq introduces dependent types which allow it to be used to be used as a proof assistant.
However, this extension to the type system requires Coq to enforce the following two restrictions:
\begin{enumerate}
  \item all pattern matching must be exhaustive and
  \item all functions must terminate on all inputs.
\end{enumerate}
In consequence, there is no partiality, no infinite data structures and Coq is not \textit{Turing complete}.
Satisfying the termination requirement of Coq presents one of the mayor challenges for the implementation of our compiler.
The next section, presents an approach to model partiality in Coq.
