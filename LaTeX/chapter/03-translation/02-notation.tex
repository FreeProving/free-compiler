\section{Notation} \label{sec:translation:notation}
\subsection{Notation for Translation Rules}
The translation rules presented in this chapter are based on the work by \cite{Abel:2005}.
We are also going to adopt their notation and write
\[
  \toCoq{H} = G
\]
to express that the Haskell language construct $H$ (e.g. a type, expression or declaration) should be converted to the corresponding Gallina language construct $G$ (e.g. a term or sentence).

\subsection{Naming conventions}
In the translation rules we make extensive use of placeholders for identifiers, expressions or types.
For the sake of readability those placeholders will not always be introduced explicitly.
Instead we rely on naming conventions outlined in this section to determine what the placeholder stands for.
If there are multiple placeholders of the same kind, an index will be appended.

The symbols $\tau$, $T$, $\alpha$ are used in place of type expressions, constructors and variables respectively.
Data type and type synonym declarations will usually be named $D$ or $S$.
Analogously $e$, $C$ and $x$ are used for regular expressions, constructors and variables.
Function declarations are called $f$ by convention.
A lower case $c$ is used as placeholder for the name of a data constructor in Coq.
It should usually differ from a capital $C$ only in that its first letter is converted to lower case.
E.g. if $C_i = \texttt{Foo}$ for some index $i$, then $c_i$ will be \coq{foo}.

We are also using $\Shape$ and $\Pos$ as placeholders for the arguments of the free monad.
They are passed explicitly as parameters of generated Gallina sentences.

\subsection{Notation for renamed identifiers}
Not all Haskell identifiers are valid Coq identifiers and need to be renamed if necessary.
For example \haskell{with} could be used in Haskell as the name for a function or variable, but not in Coq as it is a keyword.

Similarly Haskell allows types and constructors to have the same name because their namespaces are separated.
But Coq is a dependently typed language, that is types in Coq can contain values.
Therefore, Coq constructors would conflict with same-named types and may need to be renamed as well.

Details on how identifiers are actually being renamed will be given in \autoref{chp:implementation}.
In this chapter we will simply write $i'$ for the renamed version of a Haskell identifier $i$.
For short we write $e'$ or $\tau'$ for an expression $e$ or type expressions $\tau'$ in which all identifiers have been renamed appropriately.
