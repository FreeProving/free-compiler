\section{Notation} \label{sec:translation:notation}

\subsection{Notation for Translation Rules}

The translation rules presented in this chapter are based on the work by \cite{Abel:2005}.
We are also going to adopt their notation and write

\[
  \toCoq{H} = G
\]

to express that the Haskell language construct $H$ (e.g. a type, expression or declaration) should be converted to the corresponding Gallina language construct $G$ (e.g. a term or sentence).

\subsection{Notation for renamed identifiers}

Not all Haskell identifiers are valid Coq identifiers and need to be renamed if necessary.
For example \haskell{with} could be used in Haskell as the name for a function or variable, but not in Coq as it is a keyword there.

Similarily Haskell allows types and constructors to have the same name because their namespaces are separated.
But Coq is a dependently typed language, that is types in Coq can contain values.
Therefore, Coq constructors would conflict with same-named types and may need to be renamed as well.

Details on how identifiers are actually being renamed will be given in \autoref{chp:implementation}.
In this chapter we will simply write $i'$ for the renamed version of a Haskell identifier $i$.
For short we write $e'$ or $\tau'$ for an expression $e$ or type expressions $\tau'$ in which all identifiers have been renamed appropriatly.
