\chapter{Configuration File Format} \label{appendix:config}
As mentioned in \autoref{sec:implementation:base-library}, there is a configuration file in the root directory of the base library that configures the names of predefined data types and operations.
The configuration file format is Tom's Obvious, Minimal Language ("TOML")\footnote{\url{https://github.com/toml-lang/toml}}.
TOML's syntax looks similar to INI files but is standardized in contrast to INI.
TOML aims to be more readable and to easier to comprehend that YAML and is superior to JSON as it allows for the usage of comments~--~which are an essential part of configuration files.

The TOML document in the \path{env.toml} file must contain three arrays of tables \toml{types}, \toml{constructors} and \toml{functions}.
Each table in these arrays defines a type, constructor or function, respectively.
The expected contents of each table are described below.

\paragraph{Types}
The tables in the \toml{types} array must contain the following key/value pairs:
\begin{itemize}
  \item \toml{haskell-name} (String) the Haskell name of the type constructor.
  \item \toml{coq-name} (String) the identifier of the corresponding Coq type constructor.
  \item \toml{arity} (Integer) the number of type arguments expected by the type constructor.
\end{itemize}

For example, the following entry defines the pair data type.
\begin{minted}{toml}
  [[types]]
    haskell-name = "(,)"
    coq-name     = "Pair"
    arity        = 2
\end{minted}

\paragraph{Constructors}
The tables in the \toml{constructors} array must contain the following
key/value pairs:
\begin{itemize}
  \item \toml{haskell-type} (String) the Haskell type of the data constructor.
  \item \toml{haskell-name} (String) the Haskell name of the data constructor.
  \item \toml{coq-name} (String) the identifier of the corresponding Coq data constructor.
  \item \toml{coq-smart-name} (String) the identifier of the corresponding Coq
    smart constructor.
  \item \toml{arity} (Integer) the number of arguments expected by the data constructor.
\end{itemize}

For example, the following entry defines the data constructor for pairs.
\begin{minted}{toml}
  [[constructors]]
    haskell-type   = "a -> b -> (a, b)"
    haskell-name   = "(,)"
    coq-name       = "pair_"
    coq-smart-name = "Pair_"
    arity          = 2
\end{minted}

\paragraph{Functions}
The tables in the \toml{functions} array must contain the following
key/value pairs:
\begin{itemize}
  \item \toml{haskell-type} (String) the Haskell type of the function.
  \item \toml{haskell-name} (String) the Haskell name of the function.
  \item \toml{coq-name} (String) the identifier of the corresponding Coq function.
  \item \toml{arity} (Integer) the number of arguments expected by the function.
\end{itemize}

For example, the following entry defines the equality test for integers.
\begin{minted}{toml}
  [[functions]]
    haskell-type = "Integer -> Integer -> Bool"
    haskell-name = "=="
    coq-name     = "eqInteger"
    arity        = 2
\end{minted}
