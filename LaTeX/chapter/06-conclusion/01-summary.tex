\section{Summary and Results} \label{sec:conclusion:summary}
The goal of this thesis was to develop a compiler for the monadic translation for effectful Haskell programs to Coq.
First, we have familiarized ourselves with Coq and presented the approach by \cite{Dylus:2018} for modeling effectful Haskell code in Coq.
Next, we formalized the monadic translation for a subset of Haskell based on the translation from Haskell to Agda by \cite{Abel:2005}.
The resulting translation rules were implemented in Haskell with the help of third party libraries.
We extended and adapted an existing prototypical implementation by \cite{Jessen:2019} for our implementation.

As demonstrated by our case study in \autoref{chp:case-study}, our compiler generates code which is most often accepted by Coq without user intervention.
Minor modifications are sometimes needed due to the lack of type inference.
The extension for the translation of QuickCheck properties presented in \autoref{sec:translation:quickcheck} helps the user to prove properties by generating templates of theorems from Haskell code.
Implicit assumptions of the Haskell code need to be modeled by the user manually in order to prove some properties.
Parts of this process could be automated in the future.

In total, we are confident that the compiler presented in this thesis provides a solid basis for future research and the verification of effectful Haskell programs.
Nevertheless, our implementation is far from perfect.
We present some ideas for future improvements and extensions of our work in the next section.
