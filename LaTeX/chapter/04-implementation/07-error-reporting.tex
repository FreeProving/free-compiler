\section{Error Reporting} \label{sec:implementation:error-reporting}
Errors can occur in all stages of the compilation process.
For example, there could be:
\begin{itemize}
  \item an IO error while loading the input file,
  \item a syntax error when parsing the input file's content,
  \item an error due to the usage of unsupported Haskell features during the simplification of the AST or
  \item an error due to the usage of an undefined identifier during the conversion to Coq.
\end{itemize}
One way to report errors in Haskell is by using the \haskell{error} function which prints an error message and terminates the program immediately.
However, we do not want our compiler to crash but rather print the error message in a user-friendly manner and terminate gracefully.
Furthermore, there are non-fatal kinds of messages, i.e., hints and warnings, that we would like to report to the user.
For example, the user should be informed when an identifier is renamed during the conversion to Coq.

Therefore, we model the entire compilation process as a monadic computation based on the \haskell{Maybe} and \haskell{Writer} monads.
Our monad collects reported messages during the computation and cancels the computation if a fatal error occurs.
When the computation returns to the compiler's main loop, the collected error messages are printed to the console via the \haskell{IO} monad.
We are printing messages in a format based on the GHC's message format.
For example, the following program
\begin{minted}[linenos]{haskell}
module Test where

head :: [a] -> a
head (x : xs) = x
\end{minted}
causes the following error message to be reported.
\begin{minted}{text}
Test.hs:4:6: error:
    Expected variable pattern.
   |
 4 | head (x : xs) = x
   |      ^^^^^^^^
\end{minted}
