\section{Partiality Analysis} \label{sec:implementation:partiality-analysis}
In addition to sorting function declarations during the dependency analysis, the function dependency graph can also be used to identify partial functions.
As discussed in \autoref{sec:translation:func-decl:partial}, partial functions can be identified by adding two nodes additional nodes to the dependency graph for \haskell{undefined} and \haskell{error}.
However, we cannot simply add an entry for \haskell{undefined} and \haskell{error} to the list of triples from the previous section since there is no AST node for them that could go into the first components.

Therefore, we first have to look through the list of triples for the names of functions that refer to the identifiers \haskell{error} or \haskell{undefined}, i.e., identify directly partial functions.
We can use \haskell{Data.Graph} to find the names of all indirectly partial functions, i.e., all nodes from which a path to a directly partial function exists within the dependency graph.
