\chapter{Introduction} \label{chp:introduction}
% TODO motivate verification of Haskell programs.

Interactive theorem provers like Coq allow their users to reason about functional programs.
In the programming language provided by the proof assistant, the user first writes a functional program and specifies properties that should be satisfied by the program.
The interactive theorem prover assists the user in proving the specified properties.
Usually, a verified program can be extracted in another programming langauge~--~e.g., Haskell~--~once the correctness of the program has been verified with respect to its specification.
This approach is not applicable to existing Haskell code, though.
To reason about existing Haskell programs in Coq, we need to convert Haskell code to the specification language used by Coq.

The \texttt{hs-to-coq} compiler developed by \cite{SpectorZabusky:2017} performs such a translation from Haskell to Coq.
Their compiler is primarily designed to translate total Haskell programs.
The restriction to total Haskell programs is inherited from Coq's requirement that all functions must be total.
Unfortunately, most real Haskell programs are not total.
In Agda~--~a proof assistant similar to Coq~--~the same limitation exists.
\cite{Abel:2005} presented an approach for the translation from Haskell to Agda that does not require the original Haskell programs to be total.
The translation of partial functions is realized through a process known as \textit{monadic translation}.
By performing such a monadic translation, \cite{Jessen:2019} implemented a prototype for a compiler from potentially partial Haskell programs to Coq.

In this thesis we want to build a compiler for the monadic translation from Haskell to Coq as well.
In addition to partiality, we would like to model other \textit{ambient effects} \citep{Christiansen:2019} that can occur in Haskell programs but not in Coq.
We employ an approach presented by \cite{Dylus:2018} to achieve an \textit{effect-generic} monadic translation.

The remainder of this thesis is structured as follows.
We begin in \autoref{chp:preliminaries} with the preliminaries of this thesis.
This includes among others an introduction to the Coq proof assistant and an explanation of the principles behind the effect-generic monadic translation.
In \autoref{chp:translation} the translation rules for the conversion from Haskell to Coq are formalized and in \autoref{chp:implementation} we present the actual implementation of our compiler.
The translation rules and their implementation is evaluated by performing a case study in \autoref{chp:case-study}.
The last chapter discusses and summarizes our implementation as well as the results of our case study.
Furthermore, we provide an outlook on future improvements and extensions of our compiler.
