\documentclass[10pt]{book}
\usepackage[
          language=english
        , theorems=numbersfirst
        , paper=a4paper
        , bibresource={thesis.bib}
        ]{ifiseries}

% <++PERHAPS CHANGE NAME OF BIBLIOGRAPHY RESOURCE ON LINE ABOVE++>
% <++USE language=german IF YOUR DOCUMENT WILL BE IN GERMAN ++>

\usepackage{minted}

\begin{document}

\frontmatter

\studtitlepage%
{<++TITLE++> \\[.1em]
<++OVER TWO LINES++>}%
{<++SUBTITLE++>}%
{<++NAME++>}%
{<++THESIS TYPE++>}%
{<++MONTH AND YEAR++>}%
{Programming Languages and Compiler Construction}%
% in case of a german thesis you should use
% {Programmiersprachen und \"Ubersetzerkonstruktion}%
{Michael Hanus}
\cleardoublepage
\eidesstatt

\chapter*{Abstract}
<++ABSTRACT++>

\section*{Acknowledgements}
<++ACKNOWLEDGEMENTS++>

\tableofcontents
\mainmatter

\chapter{Introduction}
\chapter{Functional Logic Programming}
\cite{hanus2013functional} gives a detailed overview on the functional
logic programming language Curry.

\begin{minted}[mathescape,
               numbersep=5pt,
               frame=lines,
               framesep=2mm]{haskell}
--  In a set-based semantics, the `?`-operator can be understood as $\cup$.
(?) :: a -> a -> a
x ? y = x
x ? y = y

coin :: Bool
coin = 1 ? 2
\end{minted}
\chapter{<++THIRD CHAPTER++>}
\chapter{<++FOURTH CHAPTER++>}
\chapter{Conclusion and Future Work}

\appendix
\chapter{<++FIRST CHAPTER OF APPENDIX++>}
\chapter{<++SECOND CHAPTER OF APPENDIX++>}

\backmatter
\tocbibliography
\end{document}
