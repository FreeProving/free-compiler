\section{Future Work} \label{sec:conclusion:future-work}

% - Integrate Malte's tool.
The subset of Haskell supported by our compiler is very restrictive at the moment.
To eliminate the limitation on pattern matching, a pattern matching compiler library\footnote{\url{https://git.informatik.uni-kiel.de/stu204333/placc-thesis}} that is currently being developed by \aspellIgnore{Malte Clement}, could be integrated into our compiler in the future.
% - Implement `newtype` and bang patterns.
How Haskell features involving strictness, e.g., \haskell{newtype} declarations and bang patterns, could be modeled in Coq is described by \citet[pp.~132-134]{Christiansen:2019}.
% - Add type inference, let/where and type classes.
Further language features that we would like to support in the future include type classes, local declarations and type inference.
% - More predefined data types and operations.
More predefined data types and operations from the Prelude should be added as well.
Due to the modular design of our Base library, we expect the extension of the Prelude to be straightforward.
% Add support for module dependencies.
The configuration file format developed for the Base library could be reused for the implementation of \haskell{import} declarations as well.

% - Improve translation of recursive functions.
Furthermore, improvements need to be made to the translation of recursive functions.
Even though the detection of decreasing arguments is very complicated already, its limitations greatly reduce the user's flexibility.
In addition, the current translation scheme also does not work very well in conjunction with recursive higher-order functions.

% - Prove correctness of translation rules.
While we have tried to convince ourselves in \autoref{chp:translation} that the translation rules are most likely correct, we have not actually proven their correctness in this thesis.
Before definitive conclusions can be drawn from proofs about translated programs, it remains to be formally verified that our translation rules model the behaviour of the original program correctly.
% - Model sharing in Coq.
One aspect of Haskell's semantics which we know is not reflected by the generated Coq code is sharing.
Effects are triggered in our implementation through the use of bind operators.
If the value is bound twice, the effect is triggered twice.
In Haskell sharing prevents a value from being evaluated twice, therefore every effect is triggered at most once.
In case of partiality, we cannot distinguish between an effect being triggered once or twice \cite[p.~131]{Christiansen:2019}.
However, we can use the free monad to model a logging effect as provided by the function \haskell{trace} in Haskell.
In contrast to partiality, we can distinguish an implementation with and without sharing for this effect since in the absence of sharing, the same message may be logged multiple times \cite[p.~131]{Christiansen:2019}.
Therefore, the compiler must be adapted to model sharing before it is truly effect-generic.
